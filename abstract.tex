
\documentclass[11pt,reqno,twocolumn]{article} % use larger type; default would be 10pt

\usepackage{multicol}
\usepackage{my_packages}
\usepackage{tikz_packages}
\usepackage{float}
\tdplotsetmaincoords{60}{125} % view angle in spherical coordinates
\renewcommand{\thesection}{\Roman{section}} 
\renewcommand{\thesubsection}{\thesection.\Roman{subsection}}

\title{Simultaneous landing and mapping on Asteroid Itokawa}
\author{Shankar Kulumani}
% \thanks{PhD Candidate, Department of Mechanical and Aerospace Engineering, Email:~\href{mailto:skulumani@gwu.edu}{skulumani@gwu.edu}}


\date{} % Activate to display a given date or no date (if empty),
         % otherwise the current date is printed 

\begin{document}
\maketitle
\subsubsection*{Motivation}
% Motivation for missions/studying asteroids
Small solar system bodies, such as asteroids and comets, are of significant interest to the scientific community.
These small bodies offer great insight into the early formation of the solar system.
Of particular interest are those near-Earth asteroids (NEA) which inhabit heliocentric orbits in the vicinity of the Earth.
These easily accessible bodies provide attractive targets to support space industrialization, mining operations, and scientific missions.
Furthermore, these asteroids are of keen interest for more practical purposes.
The recent meteor explosions in  2002 over Tagish Lake, Canada or over Chelyabinsk, Russia in 2013 are clear evidence of the risk of asteroid impacts on the Earth.
These asteroids, which released an energy equivalent to \SI{5}{\kilo\tonne} of TNT, are estimated to strike the Earth on average every year~\cite{brown2002}.
Larger bodies, such as the \SI{60}{\meter} object that exploded over Tunguska, Russia in 1908, release the energy equivalent to \SI{10}{\mega\tonne} of TNT and will occur on average every \num{1000} years.
In spite of the significant interest in asteroid deflection, and the extensive research by the community, the operation of spacecraft in their vicinity remains a challenging problem.

\subsubsection*{Research Question}
% describe our approach in this paper
In this work, we develop a orbit and landing scheme for spacecraft on an asteroid.
The main objective is to construct the coupled equations of motion of a rigid spacecraft about an asteroid.
This accurate dynamic model is then used to derive a nonlinear controller for the tracking of a landing trajectory.
In contrast to much of the previous work, we explicitly consider the gravitational coupling between the orbit and attitude dynamics.
In addition, we utilize a polyhedron potential model to represent the shape of the asteroid, which results in an exact closed form expression of the gravitational potential field~\cite{werner1994,werner1996}.

In order to determine the shape of the asteroid, we model a laser ranging sensor (LIDAR) on a maneuvering spacecraft.
The LIDAR is able to provide depth measurements of the surface of the asteroid.
Given a set of depth measurements it is possible to compute the shape, and hence gravitational potential of the asteroid.

In short, this paper presents a nonlinear controller for the coupled motion of a spacecraft around an asteroid.
The dynamics are developed on the nonlinear manifold of rigid body motions, namely the special euclidean group.
This intrinsic geometric formulation accurately captures the coupling between the orbit and attitude dynamics. 
Due to the relative size of the spacecraft as compared to the orbital radius, there is a significant gravitational moment on the spacecraft. 
Through the use of the polyhedron gravitational model we ensure an accurate representation of the gravitational moment on the spacecraft throughout all phases of flight. 
Furthermore, we present a nonlinear controller developed on the special euclidean group which asymptotically tracks a desired landing trajectory. 

\subsection*{Mathematical Formulation}\label{se:mathematical_problem}
In this paper, we consider the landing of a dumbbell model of a spacecraft onto an asteroid.
The dumbbell spacecraft consists of two masses connected by a massless rod and is a well-known representation of a multi body spacecraft.
Furthermore, the dumbbell model captures the important interactions of the coupling between orbital and attitude dynamics. 
As a result, this simple model is useful to capture the main characteristics of a wide variety of spacecraft configurations.
Typically, spacecraft have mass concentrated in a central structure, referred to as the bus, which houses the command and control system, actuators, fuel, sensors etc. 
In addition, comparatively light-weight solar panels extend from the bus to provide electrical energy from solar radiation. 
As a result, the distributed mass of the spacecraft is captured with the dumbbell representation.
In this section, we briefly review the polyhedron potential model and then present the derivation of the coupled dynamics of a dumbbell spacecraft about an asteroid.

Can define the gravity field given only the shape of the asteroid

\subsection*{Technical Approach}

Determine the shape of the asteroid

Simulate a depth sensor around the asteroid

Using the depth measurements to create a shape model

Shape model allows for the definition of the gravity field
\bibliographystyle{plain}
\bibliography{library.bib}

\end{document}
